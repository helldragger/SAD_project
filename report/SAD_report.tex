\documentclass{scrreprt}

\usepackage[utf8]{inputenc}
\usepackage{graphicx}
\usepackage[french]{babel}
\usepackage{multirow}
\usepackage[dvipsnames]{xcolor}
\usepackage[allbordercolors=white]{hyperref}
\usepackage{mdframed}
\usepackage[OT1]{fontenc}

\usepackage[bottom=2cm,footskip=8mm]{geometry}

\newmdenv[
rightline=false,
topline=false,
bottomline=false,
backgroundcolor=BurntOrange!5,
fontcolor=BrickRed,
linecolor=Red,
linewidth=1pt]{problem}


\newmdenv[
rightline=false,
topline=false,
bottomline=false,
backgroundcolor=ForestGreen!5,
fontcolor=OliveGreen,
linecolor=Green,
linewidth=1pt]{result}


\newmdenv[
rightline=false,
topline=false,
bottomline=false,
backgroundcolor=Cyan!5,
fontcolor=Blue,
linecolor=NavyBlue,
linewidth=1pt]{info}

%Page style

\pagestyle{headings}
\pagenumbering{arabic}
\KOMAoption{headsepline}{true}
\KOMAoption{footsepline}{true}
\KOMAoption{twoside}{false}
\KOMAoption{abstract}{false}
\KOMAoption{DIV}{10}


%Title page


\titlehead{\centering {\Large \bfseries RAPPORT DE PROJET}\\
\itshape{ 2ème année de licence d'informatique}}
\subject{}
\title{\Huge \bfseries SAD PROJECT}
\author{Réalisé par \\Christopher JACQUIOT \& Morine PINOT}
\date{}
\publishers{\includegraphics[width=0.5\textwidth]{pics/logo_long.png}}

\begin{document}
    \maketitle
    \tableofcontents

    \chapter{Introduction}

    \paragraph{Quel est ce projet?}
    Le but de ce projet est d'expérimenter un peu en intelligence artificielle
    par le biais d'un jeu d'infection de graphe.
    Le principe de la simulation est simple, un attaquant et un defenseur
    infectent et protègent respectivement des serveurs tour à tour.
    Le but du défenseur est de garder le plus gros sous réseau de serveurs
    non infectés possible tandis que l'attaquant doit veiller à ce qu'il n'en
    reste pas grand chose en fin de partie.


    \paragraph{Qui sommes nous?}
    Nous sommes deux étudiants de L2 informatique en études à l'université de
    Caen:\\


    \begin{itemize}
        \item 21501785: Jacquiot Christopher alias Helldragger
        \item 21501666: Pinot Morine alias PINOTMorine
    \end{itemize}

    \paragraph{Comment s'y retrouver?}
    La table des matières commence à la seconde page et chaque élément de la
    table est un lien cliquable pouvant vous menez où vous le souhaitez.

    \paragraph{Pourquoi un intitulé "SAD PROJECT"}
    Car l'anglais fait partie de notre langue de travail de tous les jours et
    les initiales de la matière Sécurité et Aide à la Décision se prêtent aux
    jeux de mots!

    \begin{info}
        Ce rapport se veut non-exhaustif concernant toutes les optimisations et
        choix de design en faveur d'un volume plus court. Il est donc moins
        fourni qu'initialement prévu! Si vous souhaitez une version plus
        complète, n'hésitez pas à demander plus de détails!
    \end{info}


    \part{Fonctionnalités de base}

    %\chapter{Gestion de la partie - Structure des données du jeu}

    %\section{Nécessités}

    %\paragraph{}

    %\section{Problème}

    %\section{Approches possibles}

    %\section{Approche utilisée}

    %\section{Remarques sur les résultats obtenus}

    %\section{Pistes d'amélioration}



    %Simulation mode
    %Affichage du jeu
    %Generation de map
    %Heuristique de l'IA
    %Algorithme de l'IA



    \chapter{Gestion de la partie - Simulation des mouvements}

    \section{Nécessités}

    \paragraph{Gestion des parties simulable}
    Afin de pouvoir utiliser pleinement notre algorithme min max pour
    l'intelligence artificielle prévue, nous allons avoir besoin de diminuer
    un maximum l'impact d'un tour de jeu en temps de calcul pour laisser plus
    de temps au calcul du meilleur mouvement.

    \section{Problème}

    \paragraph{L'utilisateur doit pouvoir comprendre ce qu'il se passe}
    Avoir un tour de jeu quasi instantané est gênant quand il s'agit de
    laisser l'utilisateur avoir une vision claire de ce qu'il se passe.
    Nous avons donc besoin d'un temps d'affichage entre chaque mouvement pour
    permettre au joueur de cocmprendre le deroulement de la partie, tout en
    laissant l'IA aller aussi vite que possible pour que sa recherche de
    mouvement soit la plus efficace et rapide possible.

    \begin{problem}
        Comment allier vitesse d'execution et comprehensibilité du jeu pour
        l'utilisateur?
    \end{problem}

    \section{Approches possibles}

    \paragraph{Deux gestion de parties séparées}
    Nous pourrions avoir deux types d'objet Game séparés, l'un
    spécialement dédié a la simulation et optimisé autant que possible en
    vitesse d'execution, a coté d'un autre objet Game qui contiendrait
    l'interface utilisateur, ainsi que les temps de latence nécessaires entre
    chaque tour.
    Cette solution nécesiterais de faire de la duplication de
    code et ne serait pas optimale niveau maintenance, de plus les
    optimisations de l'un qui pourrait s'appliquer à l'autre pourraient
    passer inapercues..

    \paragraph{Mode de simulation intégré}
    Nous pourrions integrer un mode simulation activable et désactivable qui
    desactiverais ou activerais respectivement les temps d'attente et les
    mises à jour d'affichage.
    Cela nous forcerais a rendre notre objet de Game le plus optimisé
    possible tout en evitant la duplication de code.
    Cette solution semble la plus simple à implémenter et la plus efficace en
    termes de maintenance.

    \section{Approche utilisée}


    \paragraph{Mode de simulation intégré}
    Nous avons décidé d'utiliser un mode de simulation intégré, cela nous
    semble la solution la plus pertinente et utile.

    \begin{result}
        Cela nous permettra de se concentrer sur l'optimisation et la préparation
        de notre IA détaillée plus bas plus longtemps.
    \end{result}


    \section{Remarques sur les résultats obtenus}

    \paragraph{Code clair et compréhensif}
    La séparation Jeu/simulation est claire et visible dans la boucle
    principale du jeu.
    Les fonctions et variables sont courtes et au plus descriptif possible
    afin de garder une compréhension facile des algorithmes mis en place.

    \section{Pistes d'améliorations}

    \paragraph{Aucune amélioration trouvée}
    Le code me semble au plus efficace et optimisé tout en restant clair et
    compréhensible.



    \chapter{Gestion de la partie - Structure des données du jeu}

    \section{Nécessités}

    \paragraph{Représenter un graphe de serveurs}
    Nous avons comme informations à stocker un graphe non-orienté, qui est
    composé de noeuds et de liens, ainsi que les etats "infecté" ou non sur
    chacun de ces noeuds.
    Nous allons avoir besoin de modifier l'etat des noeuds et de couper des
    liens spécifiques en pleine execution.

    \section{Problème}

    \paragraph{Un accès instantané aux informations}
    Afin de pouvoir calculer facilement et rapidement les mouvements
    possibles et à réaliser, nous allons avoir besoin d'un accès aux
    informations au mieux instantané.
    Malheureusement, un accès instantané peut aussi s'avérer couteux en
    mémoire, et avec un nombre élevé de noeuds et de liens, la quantité de
    mémoire occupée par un tel systeme d'acces peut s'averer problématique.

    \begin{problem}
        Comment allier rapidité de recherche et efficacité de stockage?
    \end{problem}

    \section{Approches possibles}

    \paragraph{Une liste d'états avec une case par serveur}
    Ceci semble la solution la plus naturelle, une liste des etats de chaque
    serveur possible.
    C'est aussi une solution relativement facile à
    comprendre, l'index dans la liste representant le numero du serveur,
    l'accès à l'etat d'un serveur particulier est instantané.

    Le problème avec cette approche est lorsque nous avons besoin de
    determiner les ensembles de serveurs infectés ou non pour determiner un
    nombre de serveurs potentiels jouables pour une IA.
    Cela demande de parcourir alors n serveurs juste pour determiner les
    serveurs possibles, ce qui peut faire beaucoup d'iterations inutiles en
    debut de jeu par exemple.


    \paragraph{Deux ensembles d'etats contenant les serveurs appropriés}
    Cette solution est plus abstraite mais aussi plus interessante: l'etat de
    chacun de nos serveurs ne peut être que infecté (true) ou non infecté
    (false).

    Cela nous permet de prendre le problème a l'envers: au lieu de lier les
    serveurs à leur état, lions les états à des ensembles de serveurs.
    Cela apporte plusieurs avantages vis a vis de la liste, comme la
    récupération instantanée de tous les serveurs jouables par un defenseur,
    l'impossibilité d'avoir de doublons indésirables parmi un ensemble de
    serveurs et amène à diverses simplifications concernant le choix au
    hasard de serveurs ou la determination des serveurs infectables.

    Cette solution utilisée par le biais de HashSet permettrais d'avoir une
    complexité constante pour l'utilisation de nos données d'état.



    \paragraph{Une matrice d'adjacence}
    Pour representer nos liens nous pourrions créer une matrice d'adjacence,
    cela permet d'avoir un acces instantané à la vérification que deux
    serveurs sont bien liées ou non.
    Temps de recherche constant en tête, cela veut aussi dire garder en
    mémoire tous les liens qui n'existent pas: sur n serveurs nous
    aurions $n^{2}$ valeurs à mémoriser, le coût en mémoire et en
    temps d'écriture au démarrage est donc relativement élevé.

    \paragraph{Une liste d'ensemble de serveurs voisins}
    Une autre approche moins rapide sur certaines recherches mais plus économe
    en mémoire serais de ne conserver que les liens existant par serveur par
    le biais d'une liste d'ensembles de voisins.
    Cette fois ci nous ne conserverions que les informations cruciales,
    economisant ainsi de la mémoire, et le temps de recherche reste
    acceptable, un hashset ayant un temps de recherche constant pour un
    element dans le cas général.


    \section{Approche utilisée}


    \paragraph{Ensembles d'états et liste d'ensemble de voisins}
    Nous finalement opté pour les ensembles d'états regroupés dans une
    variable dictionnaire à 2 valeurs pour simplifier leur accès et profiter
    des avantages de la manipulation d'ensembles.

    \begin{result}
        Cela nous as permis d'avoir acces instantanément à tous les serveurs
        pertinents pour chaque tour de defenseur, et presque instantanément pour
        l'attaquant.
    \end{result}

    \paragraph{Liste d'ensemble de voisins}
    De même, nous avons décidé d'utiliser les listes d'ensembles de voisins,
    afin de limiter les données aux plus cruciales: celles que l'on ne peut
    pas déduire par logique.

    \begin{result}
        Cette approche nous as aussi servi pour pouvoir profiter des opérations de
        la théorie des ensembles pour réaliser quelques optimisations en recherche
        de serveurs!
    \end{result}

    \section{Remarques sur les résultats obtenus}

    \paragraph{Les ensembles, c'est la vie!}
    L'utilisation d'ensembles s'est avérée etre une solution très élégante,
    pratique et necessitant peu de code, laissant peu de place à de
    potentielles erreurs et bugs.
    De plus, la théorie des ensemble est très agréable à utiliser pour
    certaines de nos fonctions comme la recuperation des serveurs
    infectables, etant juste l'ensemble de serveurs non infectés ET voisins
    de serveurs infectés, l'impossibilité de doublons nous permet aussi a
    cette étape de regrouper tous les voisins non infectés de tous les serveurs
    infectés dans un seul petit ensemble lisible par un humain.



    \part{Interface Homme-Machine}
    \chapter{Représentation visuelle d'un graphe}
    \section{Nécessités}

    \paragraph{Voir les bugs}
    Afin de pouvoir aisément trouver d'eventuels bugs et erreurs de logique,
    nous avons besoin d'avoir un moyen de se representer clairement et
    comprehensivement les données d'une simulation à chaque étape de la
    simulation.


    \paragraph{Observer divers comportements}
    Afin de pouvoir comparer diverses priorités et divers comportements
    possibles de notre intelligence artificielle, il nous faut aussi pouvoir
    observer en direct les choix et les tendances adoptées par celle ci.

    \section{Problème}

    \paragraph{Des données difficiles à visualiser de tête}
    L'un des majeurs soucis avec la représentation d'un graphe sous forme de
    données est la visualisation humaine de ces données.
    Donner les informations brutes de liens des serveurs et des etats demande
    un effort de concentration qui ne devrait pas etre necessaire pour se
    representer l'état global de la situation.

    \begin{problem}
        Comment représenter de façon comprehensible les données de la partie?
    \end{problem}

    \section{Approches possibles}

    \paragraph{Une représentation textuelle du graphe}
    Il serait possible de representer graphiquement le graphe sous forme de
    texte, cela en revanche necessiterais de limiter le nombre de liens à 3
    ou 4 par serveur pour garder le graphe possible.
    Cela nous demanderais en revanche de nous concentrer un peu trop sur
    la creation de cette interface graphique optionnelle.

    \paragraph{Une représentation graphique par librairie}
    Il serait aussi possible de faire appel à des librairies open-source de
    rendu de graphes, afin de ne pas être limité au niveau de nos données.
    Cela nous permettrais aussi de nous concentrer sur l'IA plutot que sur la
    creation d'une interface.

    \section{Approche utilisée}

    \paragraph{Une représentation graphique par librairie}
    Nous avons au final opté pour la solution rapide, afin de nous concentrer
    sur l'essentiel du projet et avons utilisé graphstream pour pouvoir voir
    le deroulement de nos parties en temps réel.
    De simples couleurs permettent de representer le niveau de danger des
    serveurs (infectés, infectable, protégés) et la reorganisation
    automatique des noeuds liés permet d'avoir une vision claire des
    stratégies utilisées!

    \begin{result}
        Un court temps de développement dédié à cette interface à beaucoup
        aidé la gestion des erreurs, voir les erreur arriver visuellement
        aide à localiser les problèmes rapidement en plus de donner une
        vision agréable du déroulement de la simulation.
    \end{result}

    \part{Intelligence artificielle}

    \chapter{Algorithme de décision}

    \section{Nécessités}

    \paragraph{Determiner le meilleur coup possible}
    Afin de gagner, nos IA doivent trouver les coups qui les avantageraient
    le plus par rapport à leur adversaire.
    Seulement, cela demande à l'IA de se projeter dans le futur de la partie
    et de voir ce qui pourrait se passer en conséquence de chacune de ses
    actions.
    Et le problème, c'est qu'il y en as beaucoup de possible apres chaque
    autre action...

    \paragraph{Le determiner rapidement}
    Afin de pouvoir voir la fin de la simulation de notre vivant, il va donc
    falloir trouver le meileur choix possible aussi vite que possible.
    Et aller vite implique de ne pas forcément tout explorer!

    \section{Problème}

    \paragraph{Jeu à informations complètes}
    L'intêret et la malédiction avec ce type de jeu, c'est qu'il
    est à information complète.
    C'est à dire que toutes les informations sont constamment disponible,
    rendant de ce fait une exploration possible des coups jouable totalement
    déterministe!
    Avec suffisamment de temps, nous pourrions explorer chacunes des suites
    de mouvements possibles du debut à toutes les fins potentielles de la
    partie pour determiner une potentielle stratégie gagnante...

    \paragraph{Beaucoup de coups possibles}
    Mais le problème se pose dans la quantité de chemins possibles.
    Pour déterminer la valeur d'un coup il nous faut voir où il peut mener,
    seulement, chaque coup donne une quantité conséquente de nouveaux coups à
    parcourir, et ce à chaque coup parcouru!
    La quantité de temps nécessaire pour tout parcourir deviens très vite
    exponentiel pour un jeu de ce genre..

    \begin{problem}
        Comment donc éviter de perdre trop de temps à trouver une solution
        "suffisante"?
    \end{problem}

    \section{Approches possibles}

    \paragraph{Approche MinMax}
    Nous pourrions utiliser l'approche minmax, avec une profondeur de coups
    découvert limités.
    L'avantage de cette approche est sa simplicité: nous parcourrons toutes
    les possibilités de chaque coup possible jusqu'à un certain point, en
    considerant que l'adversaire joue toujours parfaitement pour determiner
    la valeur de notre prochain coup.
    Le désavantage en revanche, est que nous parcourrons quand meme toutes
    les possibilités engendrées par chaque coup, ce qui ralentis la recherche
    considerablement à profondeur élevée.

    \paragraph{Approche NégaMax}
    Cette solution est l'équivalente à minmax sauf en un point: L'élégance du
    pseudocode.
    Là où min max nécessite de dupliquer son code afin d'alterner entre
    fonctions minimum et maximum, ici nous pourrions tirer parti du fait que
    le minimum de deux nombres positif est equivalent au maximum de ces memes
    nombres au négatif, et ainsi retirer les fonctions minimum en alternant
    juste le signe des scores évalués.
    Ce qui rends quand meme plus propre!

    \paragraph{Elagage AlphaBeta}
    Le probleme avec les deux solutions au dessus est le fait d'explorer
    TOUTES les solutions possibles, incluant des coups risqués, mais surtout
    beaucoup de coups potentiellement inutiles, au prix de précieux temps de
    calcul.
    L'élagage alpha beta tends à regler ce probleme en ajoutant des
    conditions selon lesquelles tel ou tel arbre de possibilités pourrait
    être ignoré par la recherche, pour se focaliser sur de plus prometteurs.
    L'avantage de cet élagage est qu'il est relativement simple à comprendre:
    Ne pas continuer à explorer si un meilleur chemin potentiel a déjà été
    trouvé, au risque d'ignorer des solutions au premier abord risquées mais
    stratégiquement gagnantes cachées parmi beaucoup de mauvais choix possibles.
    De plus il est facilement améliorable en reglant alpha et beta plus
    finement pour tirer de meilleurs gains de performances de cet algorithme.
    Son seul véritable désavantage à ma connaissance en revanche, c'est qu'il
    y a mieux...

    \paragraph{Elagage Monte-Carlo}
    Il y a mieux que alpha-beta.
    Il y a Monte-Carlo.
    Le concept est plus abstrait à comprendre, les maths derriere sont aussi
    plus complexes, mais aussi superbes et nettement plus élégantes que les
    force brutes dont on parlais jusqu'à présent.
    Le principe est pourtant simple, si nous tirions toujours à l'aléatoire
    parmi un ensemble de possibilités, sur un temps infini, nous finirions
    par éventuellement avoir exploré toutes les possibilités qui y existent!
    Le résultat de ce genre d'approche revenant donc au résultat d'un min max
    sans devoir analyser branche par branche les resultats de chaque choix!
    L'avantage de cette solution est clair: Au lieu de se focaliser sur tous
    les resultats possibles d'un choix, choisissons un sous ensemble
    aléatoire parmi nos choix, puis un nouveau sous ensemble parmi les
    nouveaux choix possibles, etc etc; cela nous permet d'explorer beaucoup
    plus loin dans le déroulement de la partie sans perdre trop de temps, et
    ainsi de se focaliser sur les conséquences sur un futur plus lointain de la
    partie que de se focaliser sur les conséquences immédiates de notre coup!
    L'unique problème de cette solution est qu'on n'est pas censé s'en servir
    ou meme pouvoir l'utiliser pour ce TP, ce qui est tout de même super
    triste..

    \section{Approche utilisée}

    \paragraph{Approche MinMax avec élagage AlphaBeta}
    Nous avons au final opté pour une solution classique, le MinMax avec
    élagage Alpha-Beta.
    Bien que n'étant ni particulièrement élégante ou optimisée, cela nous as
    permis de faire de légères modifications aisément pour expérimenter
    quelques modifications.

    \begin{result}
        C'est une solution à la fois simple à implémenter et assez modulaire
        pour permettre des experiences sans trop le rendre illisible.
    \end{result}

    \part{Expérimentations}

    \chapter{}

    %% TEMPLATE pour chapitre

    %\chapter{PARTIE}

    %\section{Nécessités}

    %\paragraph{}

    %\section{Problème}

    %\paragraph{}

    %\begin{problem}

    %\end{problem}

    %\section{Approches possibles}

    %\paragraph{}

    %\section{Approche utilisée}

    %\paragraph{}

    %\begin{result}

    %\end{result}

    %\section{Remarques sur les résultats obtenus}

    %\paragraph{}

    %\section{Pistes d'amélioration}

    %\paragraph{}


\end{document}